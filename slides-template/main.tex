\documentclass{beamer}

\mode<presentation> { \usetheme{gruvbox} }
\setbeamerfont{frametitle}{size=\huge}

\usepackage{graphicx} % Allows including images
\usepackage{booktabs} % Allows the use of \toprule, \midrule and \bottomrule in tables
%\usepackage{listings}             % Include the listings-package
\usepackage{minted}
\usepackage{tikz}
\usepackage{drawstack}
\usetikzlibrary{calc,shapes.callouts,shapes.arrows,chains,positioning,fit,shapes, arrows.meta, arrows}
\usepackage{verbatimbox}
\usepackage{tcolorbox}

\usemintedstyle{gruvbox} 

\newcommand{\pointthis}[2]{
        \tikz[remember picture,baseline]{\node[anchor=base,inner sep=0,outer sep=0]%
        (#1) {\underline{#1}};\node[overlay,rectangle callout,%
        callout relative pointer={(0.2cm,0.7cm)},fill=green!50] at ($(#1.north)+(-.5cm,-1.4cm)$) {#2};}%
        }%

%----------------------------------------------------------------------------------------
%	TITLE PAGE
%----------------------------------------------------------------------------------------

\title[Example RE Slide Deck]{\huge \textbf{Example RE Slide Deck:} \\ Some Subtitle - Code in ~C} % The short title appears at the bottom of every slide, the full title is only on the title page

\author{Rowan Hart} % Your name
\date{\today} % Date, can be changed to a custom date

\begin{document}

\begin{frame}
\titlepage % Print the title page as the first slide
\end{frame}


\begin{frame}
\frametitle{Overview} % Table of contents slide, comment this block out to remove it
\tableofcontents % Throughout your presentation, if you choose to use \section{} and \subsection{} commands, these will automatically be printed on this slide as an overview of your presentation
\end{frame}

\section{An Example Section of the Presentation}

\begin{frame}
    \frametitle{An example first slide}
    Some RE stuff:
    \begin{itemize}
        \item A thing
        \item Another thing
    \end{itemize}
\end{frame}

\section{Another Example Section of the Presentation}

\begin{frame}
    \frametitle{An example of some stack frame stuff}
    \begin{tikzpicture}[scale=.5,draw=lightred,text=invtext]
        \tiny % can also use any from https://tex.stackexchange.com/questions/107057/adjusting-font-size-with-tikz-picture
        \stacktop{}
            \startframe
                \cell{char input[32]} \cellcom{RBP - 64}
                \cell{int * \textit{c}} \cellcom{RBP - 32}
                \cell{int \textit{b}} \cellcom{RBP - 24}
                \cell{int \textit{a}} \cellcom{RBP - 16}
                \cell{Stack Canary} \cellcom{RBP - 8}
                \cell{Saved RBP} \cellptr{RBP, RSP}
                \bcell{Saved RIP} \cellcom{RBP + 8}
            \finishframe{function \\ {\tt foo ()}}
            \startframe
                \cell{Stack Canary} \cellcom{RBP - 8}
                \cell{Saved RBP} \cellptr{RBP, RSP}
                \bcell{Saved RIP} \cellcom{RBP + 8}
            \finishframe{function \\ {\tt main ()}}
        \stackbottom{}
    \end{tikzpicture}
\end{frame}


\end{document}