\documentclass{beamer}

\mode<presentation> { \usetheme{gruvbox} }
\setbeamerfont{frametitle}{size=\huge}

\usepackage{graphicx} % Allows including images
\usepackage{booktabs} % Allows the use of \toprule, \midrule and \bottomrule in tables
%\usepackage{listings}             % Include the listings-package
\usepackage{minted}
\usepackage{tikz}
\usepackage{drawstack}
\usetikzlibrary{calc,shapes.callouts,shapes.arrows,chains,positioning,fit,shapes, arrows.meta, arrows}
\usepackage{verbatimbox}
\usepackage{tcolorbox}

\usemintedstyle{paraiso-dark}

\newcommand{\pointthis}[2]{
    \tikz[remember picture,baseline]{\node[anchor=base,inner sep=0,outer sep=0]%
    (#1) {\underline{#1}};\node[overlay,rectangle callout,%
    callout relative pointer={(0.2cm,0.7cm)},fill=green!50] at ($(#1.north)+(-.5cm,-1.4cm)$) {#2};}%
}%

%----------------------------------------------------------------------------------------
%	TITLE PAGE
%----------------------------------------------------------------------------------------

\title[Cryptography Day 2]{\huge \textbf{Cryptography Day 2}} % The short title appears at the bottom of every slide, the full title is only on the title page

\author{Brandon Hernandez} % Your name
\date{\today} % Date, can be changed to a custom date

\begin{document}

\begin{frame}
    \titlepage % Print the title page as the first slide
\end{frame}


\begin{frame}
    \frametitle{Overview} % Table of contents slide, comment this block out to remove it
    \tableofcontents % Throughout your presentation, if you choose to use \section{} and \subsection{} commands, these will automatically be printed on this slide as an overview of your presentation
\end{frame}

\section{Brief Review}
\begin{frame}
    \frametitle{Brief Review}
    \begin{itemize}
        \item Classical Crypto

    \end{itemize}
\end{frame}

\section{Stream Ciphers}
\begin{frame}
    \frametitle{Stream Ciphers}
    \begin{itemize}
    	\item One-Time Pad
    \end{itemize}
\end{frame}

\section{Block Ciphers}
\begin{frame}
    \frametitle{Block Ciphers}
    \begin{itemize}
    	\item AES mode of operations
    \end{itemize}
\end{frame}

\section{Diffie-Hellman}
\begin{frame}
    \frametitle{Diffie-Hellman}
\end{frame}


\section{RSA}
\begin{frame}
    \frametitle{The RSA Cryptosystem}
\end{frame}
\subsection{The RSA Problem}
\begin{frame}
    \frametitle{The RSA Problem}
    \begin{itemize}
        \item Consider two large primes, $p$ and $q$
        \item $N = pq$
        \item Consider $e$, $m$, and $c$, where $e,c, m \in \mathbb{Z}$
        \item $C \equiv m^{e} \textrm{(mod N)}$
    \end{itemize}
\end{frame}
\subsection{Construction}
\begin{frame}
    \frametitle{Construction}
    \begin{itemize}
        \item Consider two large primes, p and q
        \item Let our modulus, $N = pq$ ($\mathbb{Z}/N\mathbb{Z}$)
        \item Our public key, $e$, where $gcd(\phi{N},e) = 1$ and $1 < e < \phi{N}$
        \item The private key, $d$, where $d*e \equiv 1 (\textrm{mod} \phi{N})$
    \end{itemize}
\end{frame}

\subsection{Encryption}
\begin{frame}
    \frametitle{Encryption}
    \begin{itemize}
        \item Given $N$, $m$, and $e$
        \item $C \equiv m^{e} (mod N)$
    \end{itemize}
\end{frame}

\subsection{Decryption}
\begin{frame}
    \frametitle{Decryption}
    \begin{itemize}
        \item Given $N$, $C$, and $d$
        \item $m \equiv C^{d} \equiv m^{e*d} (mod N)$
        \item \textbf{Remeber:} $e*d \equiv 1 (mod \phi{N})$
    \end{itemize}
\end{frame}

\subsection{Decryption}
\begin{frame}
    \frametitle{Decryption}
    \begin{itemize}
        \item Given $N$, $C$, and $d$
        \item $m \equiv C^{d} \equiv m^{e*d} (mod N)$
        \item \textbf{Remeber:} $e*d \equiv 1 (mod \phi{N})$
    \end{itemize}
\end{frame}


\section{Closing Thoughts}
\begin{frame}
    \frametitle{What's left?}
    \begin{itemize}
    	\item Elliptic Curve Cryptography
       	\item Post Quantum Cryptography
    	\begin{itemize}
    		\item Lattice-Based Cryptography
    		\item LWE
    		\item Multivariate Cryptograhy
    	\end{itemize}
    \end{itemize}
\end{frame} 

\end{document} 